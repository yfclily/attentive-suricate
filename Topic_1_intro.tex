% Options for packages loaded elsewhere
\PassOptionsToPackage{unicode}{hyperref}
\PassOptionsToPackage{hyphens}{url}
%
\documentclass[
  ignorenonframetext,
]{beamer}
\usepackage{pgfpages}
\setbeamertemplate{caption}[numbered]
\setbeamertemplate{caption label separator}{: }
\setbeamercolor{caption name}{fg=normal text.fg}
\beamertemplatenavigationsymbolsempty
% Prevent slide breaks in the middle of a paragraph
\widowpenalties 1 10000
\raggedbottom
\setbeamertemplate{part page}{
  \centering
  \begin{beamercolorbox}[sep=16pt,center]{part title}
    \usebeamerfont{part title}\insertpart\par
  \end{beamercolorbox}
}
\setbeamertemplate{section page}{
  \centering
  \begin{beamercolorbox}[sep=12pt,center]{part title}
    \usebeamerfont{section title}\insertsection\par
  \end{beamercolorbox}
}
\setbeamertemplate{subsection page}{
  \centering
  \begin{beamercolorbox}[sep=8pt,center]{part title}
    \usebeamerfont{subsection title}\insertsubsection\par
  \end{beamercolorbox}
}
\AtBeginPart{
  \frame{\partpage}
}
\AtBeginSection{
  \ifbibliography
  \else
    \frame{\sectionpage}
  \fi
}
\AtBeginSubsection{
  \frame{\subsectionpage}
}
\usepackage{lmodern}
\usepackage{amssymb,amsmath}
\usepackage{ifxetex,ifluatex}
\ifnum 0\ifxetex 1\fi\ifluatex 1\fi=0 % if pdftex
  \usepackage[T1]{fontenc}
  \usepackage[utf8]{inputenc}
  \usepackage{textcomp} % provide euro and other symbols
\else % if luatex or xetex
  \usepackage{unicode-math}
  \defaultfontfeatures{Scale=MatchLowercase}
  \defaultfontfeatures[\rmfamily]{Ligatures=TeX,Scale=1}
\fi
% Use upquote if available, for straight quotes in verbatim environments
\IfFileExists{upquote.sty}{\usepackage{upquote}}{}
\IfFileExists{microtype.sty}{% use microtype if available
  \usepackage[]{microtype}
  \UseMicrotypeSet[protrusion]{basicmath} % disable protrusion for tt fonts
}{}
\makeatletter
\@ifundefined{KOMAClassName}{% if non-KOMA class
  \IfFileExists{parskip.sty}{%
    \usepackage{parskip}
  }{% else
    \setlength{\parindent}{0pt}
    \setlength{\parskip}{6pt plus 2pt minus 1pt}}
}{% if KOMA class
  \KOMAoptions{parskip=half}}
\makeatother
\usepackage{xcolor}
\IfFileExists{xurl.sty}{\usepackage{xurl}}{} % add URL line breaks if available
\IfFileExists{bookmark.sty}{\usepackage{bookmark}}{\usepackage{hyperref}}
\hypersetup{
  pdftitle={Topic 1 - First steps},
  pdfauthor={by Vianney Denis},
  hidelinks,
  pdfcreator={LaTeX via pandoc}}
\urlstyle{same} % disable monospaced font for URLs
\newif\ifbibliography
\usepackage{color}
\usepackage{fancyvrb}
\newcommand{\VerbBar}{|}
\newcommand{\VERB}{\Verb[commandchars=\\\{\}]}
\DefineVerbatimEnvironment{Highlighting}{Verbatim}{commandchars=\\\{\}}
% Add ',fontsize=\small' for more characters per line
\usepackage{framed}
\definecolor{shadecolor}{RGB}{248,248,248}
\newenvironment{Shaded}{\begin{snugshade}}{\end{snugshade}}
\newcommand{\AlertTok}[1]{\textcolor[rgb]{0.94,0.16,0.16}{#1}}
\newcommand{\AnnotationTok}[1]{\textcolor[rgb]{0.56,0.35,0.01}{\textbf{\textit{#1}}}}
\newcommand{\AttributeTok}[1]{\textcolor[rgb]{0.77,0.63,0.00}{#1}}
\newcommand{\BaseNTok}[1]{\textcolor[rgb]{0.00,0.00,0.81}{#1}}
\newcommand{\BuiltInTok}[1]{#1}
\newcommand{\CharTok}[1]{\textcolor[rgb]{0.31,0.60,0.02}{#1}}
\newcommand{\CommentTok}[1]{\textcolor[rgb]{0.56,0.35,0.01}{\textit{#1}}}
\newcommand{\CommentVarTok}[1]{\textcolor[rgb]{0.56,0.35,0.01}{\textbf{\textit{#1}}}}
\newcommand{\ConstantTok}[1]{\textcolor[rgb]{0.00,0.00,0.00}{#1}}
\newcommand{\ControlFlowTok}[1]{\textcolor[rgb]{0.13,0.29,0.53}{\textbf{#1}}}
\newcommand{\DataTypeTok}[1]{\textcolor[rgb]{0.13,0.29,0.53}{#1}}
\newcommand{\DecValTok}[1]{\textcolor[rgb]{0.00,0.00,0.81}{#1}}
\newcommand{\DocumentationTok}[1]{\textcolor[rgb]{0.56,0.35,0.01}{\textbf{\textit{#1}}}}
\newcommand{\ErrorTok}[1]{\textcolor[rgb]{0.64,0.00,0.00}{\textbf{#1}}}
\newcommand{\ExtensionTok}[1]{#1}
\newcommand{\FloatTok}[1]{\textcolor[rgb]{0.00,0.00,0.81}{#1}}
\newcommand{\FunctionTok}[1]{\textcolor[rgb]{0.00,0.00,0.00}{#1}}
\newcommand{\ImportTok}[1]{#1}
\newcommand{\InformationTok}[1]{\textcolor[rgb]{0.56,0.35,0.01}{\textbf{\textit{#1}}}}
\newcommand{\KeywordTok}[1]{\textcolor[rgb]{0.13,0.29,0.53}{\textbf{#1}}}
\newcommand{\NormalTok}[1]{#1}
\newcommand{\OperatorTok}[1]{\textcolor[rgb]{0.81,0.36,0.00}{\textbf{#1}}}
\newcommand{\OtherTok}[1]{\textcolor[rgb]{0.56,0.35,0.01}{#1}}
\newcommand{\PreprocessorTok}[1]{\textcolor[rgb]{0.56,0.35,0.01}{\textit{#1}}}
\newcommand{\RegionMarkerTok}[1]{#1}
\newcommand{\SpecialCharTok}[1]{\textcolor[rgb]{0.00,0.00,0.00}{#1}}
\newcommand{\SpecialStringTok}[1]{\textcolor[rgb]{0.31,0.60,0.02}{#1}}
\newcommand{\StringTok}[1]{\textcolor[rgb]{0.31,0.60,0.02}{#1}}
\newcommand{\VariableTok}[1]{\textcolor[rgb]{0.00,0.00,0.00}{#1}}
\newcommand{\VerbatimStringTok}[1]{\textcolor[rgb]{0.31,0.60,0.02}{#1}}
\newcommand{\WarningTok}[1]{\textcolor[rgb]{0.56,0.35,0.01}{\textbf{\textit{#1}}}}
\setlength{\emergencystretch}{3em} % prevent overfull lines
\providecommand{\tightlist}{%
  \setlength{\itemsep}{0pt}\setlength{\parskip}{0pt}}
\setcounter{secnumdepth}{-\maxdimen} % remove section numbering

\title{Topic 1 - First steps}
\author{by Vianney Denis}
\date{}

\begin{document}
\frame{\titlepage}

\begin{frame}

\INSTALL

\end{frame}

\begin{frame}[fragile]{Download and install R}
\protect\hypertarget{download-and-install-r}{}

Go to \href{http://cran.r-project.org/bin/windows/base/}{link} for
windows system or \href{https://cran.r-project.org/}{link} for mac. Run
the installer, and follow the instructions, untick \emph{`Save version
number in registry'}

\UPDATE

\begin{block}{Check for update}

\begin{Shaded}
\begin{Highlighting}[]
\ControlFlowTok{if}\NormalTok{(}\OperatorTok{!}\KeywordTok{require}\NormalTok{(installr)) \{}
  \KeywordTok{install.packages}\NormalTok{(}\StringTok{"installr"}\NormalTok{); }\KeywordTok{require}\NormalTok{(installr)\} }
\KeywordTok{updateR}\NormalTok{()}
\end{Highlighting}
\end{Shaded}

This will start the updating process of your R installation. It will
check for newer versions, and if one is available, will guide you
through the decisions you need to make

{ It will tell you if your R version is out-of-date }

\PACKAGE

\end{block}

\begin{block}{Install and use a package}

\begin{itemize}
\tightlist
\item
  Install the package {\texttt{abc}}
\end{itemize}

\begin{Shaded}
\begin{Highlighting}[]
\KeywordTok{install.packages}\NormalTok{(}\StringTok{"abc"}\NormalTok{)}
\end{Highlighting}
\end{Shaded}

{ It will tell you if your package is updated }

To do only once, unless you remove or change of version of R

\begin{itemize}
\tightlist
\item
  Use the package {\texttt{abc}}
\end{itemize}

\begin{Shaded}
\begin{Highlighting}[]
\KeywordTok{library}\NormalTok{(}\StringTok{"abc"}\NormalTok{)}
\end{Highlighting}
\end{Shaded}

To do everytime: load your package

\begin{quote}
\emph{{\textbf{R.practice 1}: Install and load the package }}
{\texttt{vegan}}
\end{quote}

\begin{Shaded}
\begin{Highlighting}[]
\KeywordTok{install.packages}\NormalTok{(}\StringTok{'vegan'}\NormalTok{)}
\KeywordTok{library}\NormalTok{(}\StringTok{'vegan'}\NormalTok{)}
\end{Highlighting}
\end{Shaded}

\HELP

\end{block}

\begin{block}{Calling for help}

\begin{Shaded}
\begin{Highlighting}[]
\NormalTok{?median}
\CommentTok{# or}
\KeywordTok{help}\NormalTok{(median)}
\end{Highlighting}
\end{Shaded}

Help on the function {\texttt{median}}

\begin{Shaded}
\begin{Highlighting}[]
\NormalTok{?median}
\end{Highlighting}
\end{Shaded}

It will give you details on how to use the function {\texttt{median}}

\begin{Shaded}
\begin{Highlighting}[]
\NormalTok{??median}
\end{Highlighting}
\end{Shaded}

It will give you all functions with median in their description \WD

Use of working directory

\begin{Shaded}
\begin{Highlighting}[]
\KeywordTok{getwd}\NormalTok{()}
\CommentTok{#Get Working Directory}
\end{Highlighting}
\end{Shaded}

Get the location of your current working directory

\begin{Shaded}
\begin{Highlighting}[]
\KeywordTok{setwd}\NormalTok{()}
\CommentTok{#Set Working Directory}
\end{Highlighting}
\end{Shaded}

Set up a new working directory

\QUITR

\end{block}

\begin{block}{Quit R}

\begin{Shaded}
\begin{Highlighting}[]
\KeywordTok{q}\NormalTok{()}
\end{Highlighting}
\end{Shaded}

It will leave your R session, do not save your workspace

\OBJECTS

\end{block}

\begin{block}{List objects}

\begin{Shaded}
\begin{Highlighting}[]
\KeywordTok{ls}\NormalTok{()}
\CommentTok{# list}
\end{Highlighting}
\end{Shaded}

It will list all objects in memory

\begin{Shaded}
\begin{Highlighting}[]
\KeywordTok{rm}\NormalTok{(}\DataTypeTok{list=}\KeywordTok{ls}\NormalTok{())}
\CommentTok{# remove list}
\end{Highlighting}
\end{Shaded}

It will remove all objects in memory

\textbf{Tip}: Use CTRL+L to clean your screen

\RCALCULATOR

\end{block}

\begin{block}{R is a calculator}

\begin{Shaded}
\begin{Highlighting}[]
\DecValTok{3}\OperatorTok{+}\DecValTok{2} \CommentTok{# addition}
\DecValTok{3-2} \CommentTok{# substraction}
\DecValTok{3}\OperatorTok{*}\DecValTok{2} \CommentTok{# multiplication}
\DecValTok{3}\OperatorTok{/}\DecValTok{2} \CommentTok{# division}
\DecValTok{3}\OperatorTok{^}\DecValTok{3} \CommentTok{# power}
\KeywordTok{log}\NormalTok{(}\DecValTok{2}\NormalTok{) }\CommentTok{# logarithm}
\KeywordTok{exp}\NormalTok{(}\DecValTok{2}\NormalTok{) }\CommentTok{# exponential}
\NormalTok{(}\DecValTok{5} \OperatorTok{+}\StringTok{ }\DecValTok{3}\NormalTok{) }\OperatorTok{/}\StringTok{ }\DecValTok{4} \CommentTok{# define priority using () or \{\} }
\NormalTok{pi}\OperatorTok{*}\DecValTok{4} \CommentTok{# common function}
\end{Highlighting}
\end{Shaded}

R is case sensitive {\texttt{pi}} will work, whereas {\texttt{Pi}} will
not

\EDITOR

\end{block}

\begin{block}{Text editor, Rite}

Install and load package {\texttt{rite}}

\begin{Shaded}
\begin{Highlighting}[]
\KeywordTok{install.packages}\NormalTok{(}\StringTok{'rite'}\NormalTok{)}
\KeywordTok{library}\NormalTok{(}\StringTok{'rite'}\NormalTok{)}
\KeywordTok{rite}\NormalTok{()}
\KeywordTok{riteout}\NormalTok{() }\CommentTok{# CTRL+Enter to send your lines}
\end{Highlighting}
\end{Shaded}

This is simple text editor

\READINGDATA

\end{block}

\begin{block}{Reading data}

Using {\texttt{riteout}}

\begin{Shaded}
\begin{Highlighting}[]
\KeywordTok{riteout}\NormalTok{() }\CommentTok{# CTRL+Enter to send your lines}
\KeywordTok{setwd}\NormalTok{ (}\StringTok{'D:/.../Class 1'}\NormalTok{)}\CommentTok{# set your working directory}
\KeywordTok{read.table}\NormalTok{ (}\StringTok{"taiwan_coral.txt"}\NormalTok{,}\DataTypeTok{header =} \OtherTok{TRUE}\NormalTok{,}\DataTypeTok{sep=}\StringTok{"}\CharTok{\textbackslash{}t}\StringTok{"}\NormalTok{,}\DataTypeTok{dec=}\StringTok{"."}\NormalTok{)}\CommentTok{# opening a simple datasheet}
\end{Highlighting}
\end{Shaded}

{ You can also target directly your file without setting up your working
directory }

\begin{Shaded}
\begin{Highlighting}[]
\KeywordTok{read.table}\NormalTok{ (}\StringTok{"D:/.../Class 1/taiwan.txt"}\NormalTok{,}\DataTypeTok{header =} \OtherTok{TRUE}\NormalTok{,}\DataTypeTok{sep=}\StringTok{"}\CharTok{\textbackslash{}t}\StringTok{"}\NormalTok{, }\DataTypeTok{dec=}\StringTok{"."}\NormalTok{)}\CommentTok{# long version}
\KeywordTok{read.table}\NormalTok{(}\StringTok{"taiwan.txt"}\NormalTok{, }\OtherTok{TRUE}\NormalTok{, }\StringTok{"}\CharTok{\textbackslash{}t}\StringTok{"}\NormalTok{,}\StringTok{"."}\NormalTok{)}\CommentTok{# you can also shorten some part once you get use to it, but be careful when using this}
\end{Highlighting}
\end{Shaded}

{ Now you can store it in R using the operator}
{\texttt{\textless{}-}}{or} {\texttt{=}}

\begin{Shaded}
\begin{Highlighting}[]
\NormalTok{taiwan <-}\StringTok{ }\KeywordTok{read.table}\NormalTok{ (}\StringTok{"taiwan.txt"}\NormalTok{,     }\DataTypeTok{header =} \OtherTok{TRUE}\NormalTok{, }\DataTypeTok{sep=}\StringTok{"}\CharTok{\textbackslash{}t}\StringTok{"}\NormalTok{, }\DataTypeTok{dec=}\StringTok{"."}\NormalTok{)}
\end{Highlighting}
\end{Shaded}

{\texttt{read.delim}} {or} {\texttt{read.csv}} { and many others are
alternative to} {\texttt{read.table}}

{Now, you can call your data by calling your object} {\texttt{taiwan}}

\begin{Shaded}
\begin{Highlighting}[]
\NormalTok{taiwan}
\end{Highlighting}
\end{Shaded}

{CONGRATS, this is your first script in R}

\#\#\#\#\textbf{\emph{{Exercice 1b:}}} {using the script editor of R
Studio (``File\textgreater New File\textgreater R Script''"):}

\begin{block}{\emph{{- change your working directory to the one you
created for this class}}}

\end{block}

\begin{block}{\emph{{- import the file {\texttt{rairuoho.txt}} (see
below) in an object called rairuoho }}}

\emph{extra: try importing the url of the file
using:{\texttt{read.table()}}}

\end{block}

\begin{block}{\emph{{- using the help facilities, find out, how you
could read just the first lines, if needed (using the
{\texttt{read.table()}} function) }}}

\end{block}

\begin{block}{\emph{{- {\texttt{sum()}}for this table, values from the
2nd column {\texttt{rairuoho}}{\texttt{{[}}} no. line{\texttt{,}} no.
column{\texttt{{]}}} }}}

\#\#\#\#{ \textbf{Knit} your script (``File\textgreater Knit Document'')
in {\texttt{html}} format and upload it below}

\end{block}

\end{block}

\end{frame}

\end{document}
